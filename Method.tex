\section{Research Method}\label{sec:researchMethod}

This section describes the research methodology used to address the research questions.
Our data collection is based on semi-structured interviews. 
The interview protocol follows the seven-stages approach by Kvale \& Brinkmann~\cite{Kvale2015Interviews}, which covers preparation, execution and reporting. The seven stages are: Thematising, Designing, Interviewing, Transcribing, Analysing, Verifying, and Reporting. The way we used them in our study is described in the rest of this Section. 

The interview results aim at providing in-depth analysis on the current situation, as well as explore the desired (future) situation. %The results also provide the foundation for the Reference Model\todo{What is that?}~\cite{Blessing:2009:DDR:1717807}. %The goal of this section is to provide a clear description of the interview process applied in this research to support and clarify the results and present the results.

\subsection{Seven stages}

\noindent {\bf Thematising}: This phase decides on the purpose and subject matter, as well as the methods to be applied. The purpose of the interview survey is to require in-depth information on the topics Continuous Integration \& Deployment and Ecosystem \& Transparency in the automotive industry. These topics are discussed with representatives of Volvo Car Corporation and its suppliers (Tier-1 and Tier-2). % Also consultancy companies who work closely with the automotive industry are interviewed for this project.\todo{what do you mean "for this project"? delete?}

\noindent {\bf Designing}: The seven stages of the approach are designed for the study. The structure of the interview survey, transcribing details, analysis protocol \cite{saldana2015coding}, verification and reporting are discussed in detail. Beforehand, the interviewee is informed on the interview details, topics and requested permission for the purpose of recording the conversation for the transcript. By doing this beforehand, the interviewee is familiar with the topics and allows him or her to reflect on the research topics upfront.

\noindent {\bf Interviewing}: The interview has been structured as a semi-structured interview where open questions guide the interview process towards a list of goals that need to be achieved in the interview. Semi-structured interviews are very suitable for exploratory research \cite{Runeson:2009:GCR:1519313.1519324}. The interview is organized in three categories of questions: introduction, main questions and cool-off.
The {\em introduction} is meant to create context / background, and to provide and maintain an informal and interactive atmosphere. For the purpose of the research, it was necessary that the interviewee is able to explain as much as possible about her or her knowledge and experience in the own field of expertise in a limited time span. 
The {\em main part} of questionnaire is aimed at the actual research. The questions aim to guide the interview, but be dynamic and interactive at the same time. This dynamic and interactive nature of interviewing is useful for exploration and development of interesting and unexpected ideas brought up by the interviewee, which are less possible with other (more structured) methodologies, such as questionnaire surveys.
At the end of the interview, or {\em cool-off}, the interviewee is given time to ask some questions about the topics, this could be topic or themes that are not mentioned in the interview or could be important for the research. Also an appointment is made with the interviewee to verify and validate the transcript.

\noindent {\bf Transcribing:} to analyse the interview results, it is required to transcribe the interview in a clear and precise manner. The interviewee and interviewer speak English and the interview will be held in English, hence, this is also the case for the transcript. Also the analysis or coding is done in English.

\noindent {\bf Analysing:} For the analysis of the interview results, a protocol is applied to code the interview transcripts, retrieve useful information and analyze it for the research. This is done by using the method explained by Saldana \cite{saldana2015coding}. The transcript is divided into smaller and easier to code parts. This could be words, phrases, paragraphs or sections. The goal of this analysis is to find keywords or uncover themes that can be of value for the research. The findings are reported in the preliminary section of the coding table. This is done by constant comparison with the research, literature study and transcript in mind. After analysing the transcript and defining the preliminary codes, the final codes are defined based on relevance to the topics, importance, repetition or explicitly mentioned as important. The final codes are then used to draw conclusions.

\noindent {\bf Verifying:} Verification activities include ascertaining the validity, reliability and generalizability of the findings. {\em Validity} refers to the question if the study investigates what was originally intended to \cite{Kvale2015Interviews}. The purpose of the interviews is to gather from the interviewees insights and knowledge on the topics. The interview is semi-structured and the questions are intended as guidelines throughout the interview, so to safeguard validity. %The guidelines are not meant to be strict, so it allows also to meander to address other related topics. 
By applying the systematic and structured approach by Kvale and Brinkmann \cite{Kvale2015Interviews}, we safeguard the {\em reliability} of conducting, analyzing and reporting  the interview process. Accordingly, the results are considered reliable too. As for {\em generalizability}, the results of the research can be applied to other domains than the automotive industry, e.g. aviation. The most advanced cars have more, or comparable, software than (fighter) air planes\footnote{As said by Alfred Katzenbach, Director of IT management at Daimler: \url{http://spectrum.ieee.org/transportation/systems/this-car-runs-on-code}}.% Moreover, we assume that the results are general applicable with similar contexts. 

\noindent {\bf Reporting:} In this paper we report the results of the analysis and we answer the research questions RQ1, RQ2, and RQ3. 
%and RQ1b defined in the Introduction\todo{revise RQs, and should answer them all}. The results also support defining the Reference Model, which in turn will lay the foundation of the (intended) Impact Model~\cite{Blessing:2009:DDR:1717807}. The Impact Model will be of advisory nature and will describe the desired, improved situation that is expected. 
Even though the study is performed within a single OEM, the suppliers that work with Volvo Cars collaborate with many other OEMs. This gives the flavour that the results of our study will easily generalize to the automotive domain. Moreover, the way automotive OEMs work with suppliers is shared with many other domains, like for instance aviation. We then expect that the our results will be valuable also outside of the automotive domain.

%and this gives the These models\todo{plural?} are designed specifically for the domain of automotive software ecosystems. However, by abstracting the models it might be possible to apply them in other similar domains, such as aviation.

\subsection{Analysis protocol}
We conducted ten interviews with practitioners in a wide range of companies in the automotive software ecosystem. The interviews were transcribed and codified in Atlas.ti\footnote{Qualitative Research tool Atlas.ti; https://atlasti.com}. This tool is used to manage quotations and codes of transcripts. The quotations and codes describe the train of thoughts of the interviewees in order to process them into feasible datasets and to help supporting or negating propositions and research questions. All information that was relevant for the research was submitted as a quotation and connected to a code, including information that was not a direct answer to a research question or proposition. The codes were grouped into code groups to give an answer to one specific question, i.e. codes relevant to proposition 1 were submitted in code group ``Proposition 1". This is done to simplify the writing process of the findings for each individual research questions and propositions. This entire process made it easy to group all relevant quotations in an overview and to create a selective dataset. This dataset was then further analysed to retrieve the answers on the questions of every interviewee. In addition, the dataset provide extra information that could benefit the findings for the specific question. This could be extra background information, another perspective on the question or maybe start a new discussion. 

The results of this process are the findings, and the answers to the research questions and propositions.