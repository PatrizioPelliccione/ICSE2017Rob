\section{Research Method}\label{sec:researchMethod}

This section describes the research methodology and used in this work to address the research questions.
Our data collection was based on semi-structured interviews. 
The interview protocol follows the  seven-stage approach by Kvale \& Brinkmann~\cite{Kvale2015Interviews}. 
They suggest a process approach to conducting an interview. The seven-stage approach supports a consistent preparation, execution and reporting of the interview. This approach entails Thematising, Designing, Interviewing, Transcribing, Analysing, Verifying, and Reporting, as detailed in the following. The interviews provide in-depth analysis on the current situation and explore the desired situation. The results of the interview surveys are the foundation for the Reference Model\todo{What is that?}~\cite{Blessing:2009:DDR:1717807}. The goal of this section is to provide a clear description of the interview process applied in this research to support and clarify the results and present the results.

\subsection{Seven stages}

\noindent {\bf Thematising}: In this phase, the purpose and subject matter is defined, also, there are decisions made on the methods applied. The purpose of the interview surveys is to require in-depth information on the topics Continuous Integration \& Deployment and Ecosystem \& Transparency in the automotive industry. These topics are discussed with representatives of Volvo Car Corporation and its suppliers (Tier-1 and Tier-2). Also consultancy companies who work closely with the automotive industry are interviewed for this project.

\noindent {\bf Designing}: In this phase, the seven stages of the approach are designed. The structure of the interview survey, transcribing details, analysis protocol \cite{saldana2015coding}, verification and reporting are discussed in detail. Beforehand, the interviewee is informed on the interview details, topics and requested permission for the purpose of recording the conversation for the transcript. By doing this beforehand, the interviewee is familiar with the topics and allows him to prepare on the research topics. This is assumed beneficial for the results of the interview.

\noindent {\bf Interviewing}:  The interview is structured as a semi-structured interview. Instead of having specific questions for the whole interview, there were questions to guide the process and a list of goals that needed to be covered in the interview, so a discussion can arise. Semi-structured interviews are very suitable for exploratory research \cite{Runeson:2009:GCR:1519313.1519324}. The interview is divided in three categories: introduction, main questions and cool-off.
The introduction is meant to create context / background, and to provide and maintain an informal and interactive atmosphere. For the purpose of the research, it was necessary that the interviewee is able to explain as much as possible about their knowledge and experiences in their field of expertise in a short period of time. 
The main part of questionnaire is aimed at the actual research. The questions were meant to guide the interview, but it was necessary to be dynamic and interactive. This dynamic and interactive nature of interviewing is useful for exploration and development of interesting and unexpected ideas brought up by the interviewee, which could not have been possible with other (strict) methodologies, such as an interview questionnaire.
At the end of the interview or cool-off, the interviewee is given time to ask some questions about the topics, this could be topic or themes that are not mentioned in the interview or could be important for the research. Also an appointment is made with the interviewee to verify and validate the transcript.

\noindent {\bf Transcribing:} In order to be able to analyse the interview, it is required to transcribe the interview in a clear and precise manner. The interviewee and interviewer speak English and the interview will be held in English, hence, this is also the case for the transcript. Also the analysis or coding is done in English.

\noindent {\bf Analysing:} For the analysis of the interview, a protocol is applied to code the interview survey, retrieve useful information and analyse it for the research. This is done by the method explained by Saldana \cite{saldana2015coding}. The transcript is divided into smaller and easier to code parts. This could be words, phrases, sentences or sections. The goal of this analysis is to find keywords or themes that could be of value for the research. The findings are reported in the preliminary section of the coding table. This is done by constant comparison with the research, literature study and transcript in mind. After analysing the transcript and defining the preliminary codes, the final codes are defined based on relevance to the topics, importance, repetition or explicitly mentioned as important. The final codes are then used to make conclusions.

\noindent {\bf Verifying:} The activities in the verification phase include ascertaining the validity, reliability and generalizability of the findings. Validity refers to the question if the study investigates what was intended at the beginning \cite{Kvale2015Interviews}. The purpose of the interview is to obtain more insights and knowledge from the interviewee on the topics. The interview is semi-structured and the questions are intended as guidelines throughout the interview, so to safeguard validity. The guidelines are not meant to be strict, so it allows also to meander to address other related topics. By applying the systematic and structured approach by Kvale and Brinkmann \cite{Kvale2015Interviews}, the conducting, analysing and reporting of the interview process can be assumed as reliable. Hence, the results are considered reliable. The results of the research can be applied in more domains than the automotive industry, such as aviation. The most advanced cars have more or comparable software than (fighter) air planes\footnote{As said by Alfred Katzenbach, the director of information
technology management at Daimler: \url{http://spectrum.ieee.org/transportation/systems/this-car-runs-on-code}}. Moreover, we assume that the results are general applicable with similar contexts. 

\noindent {\bf Reporting:} The results of the analysis are reported in this thesis paper and will answer the research questions RQ1a and RQ1b) referred to in the introduction. These results will also support the Reference Model and this model will in return be the foundation of the (intended) Impact Model~\cite{Blessing:2009:DDR:1717807}. The Impact Model will be of advisory nature and will describe the desired, improved situation that is expected. These models are designed specifically for the domain of automotive software ecosystem, but by abstracting the models it is possible to apply them in other domains similar to the software ecosystem in the automotive industry, such as aviation.

\subsection{Analysis protocol}
For this research ten interview surveys were conducted with practitioners in a wide range of companies in the automotive software ecosystem. The interviews were transcribed and submitted in Atlas.ti \footnote{Qualitative Research tool Atlas.ti; https://atlasti.com}. This tool is used to manage quotations and codes of transcripts. The quotations and codes describe the train of thoughts of the interviewees in order to process them into feasible datasets and to help to support or negate propositions and research questions. All information that was relevant for the research was submitted as a quotation and connected to a code, including information that was not a direct answer to a research question or proposition. The codes were grouped into code groups to give an answer to one specific question, i.e. codes relevant to proposition 1 were submitted in code group "Proposition 1". This is done to simplify the writing process of the findings for each individual research questions and propositions. This entire process made it easy to group all relevant quotations in an overview and to create a selective dataset. This dataset was then further analysed to retrieve the answers on the questions of every interviewee. In addition, the dataset provide extra information that could benefit the findings for the specific question. This could be extra background information, another perspective on the question or maybe start a new discussion. The results of this process are the findings and answers to the research questions and propositions.