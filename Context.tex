\section{Context of the study}\label{sec:context}


This work is made in the context of the {\em Next-Generation Electrical Architecture (NGEA)} and {\em Next-Generation Electrical Architecture - step 2 (NGEAs)} projects, funded by Vinnova~\cite{Vinnova}. 
These projects are coordinated by Volvo Cars and involve the Chalmers University of Technology, some research centers in Sweden and many suppliers of the OEM, including Autoliv, Arccore, Combitech, Cybercom, Knowit, Prevas, \AA F-Technology, and Qamcom. The project aims to develop new software processes and proof of concepts to strengthen the competitiveness of the automotive industry in Sweden. The main objectives of the projects are to investigate (i) the transition of Volvo Cars towards continuous integration and deployment, (ii) new business models and innovative ways of working within the automotive ecosystem, and (iii) vehicles as part of system of systems. 
%Included in this project are sub projects or work packages that focus on 1) management, administration and business intelligence studies, 2) continuous deployment of architectural and development strategic viewpoints, and 3) automobiles as a system within an automotive software ecosystem \cite{Vinnova}. 
In this paper we mostly focus on point (ii) even though there will be some impact on point~(i).

The automotive ecosystem consists of cross-organizational collaborations among automotive suppliers and is characterized by relying on complex supplier networks and strong dependence on hardware and software development \cite{Patrizio2016AAF_Chalmers,Knauss14d.:towards,Broy:2006:CAS:1134285.1134292}.
The current automotive industry is closed, has strict organizational boundaries, stiff processes, established business models and a straightforward value creation~\cite{ConnectedVehicle2012}.


Nowadays, a vehicle is a {\em driving software package} as compared to the vehicles of not even ten years ago. J�rgen M�ssinger, VP for automotive systems integration at Bosch, said: ``Electronics and especially software are the main sources of automotive innovation today."~\cite{Mossinger2010SoftwareAutomotive}. The Boston Consulting Group estimates that the total costs of electronic parts will rise from 20\% of the value in a typical automobile in 2004 to 40\% in 2015. Software, instead of hardware, has become the differentiating factor of products~\cite{ConnectedVehicle2012,hbr2015hardwaresoftware,Mossinger2010SoftwareAutomotive,Broy:2006:CAS:1134285.1134292}. In the past, hardware was the differentiator between companies and their products. This evolution of the automotive industry, illustrated by the exponential increase of software, creates new challenges regarding software integration, development, deployment, and maintenance. Therefore, its development needs to support the related integration and evolution of time~\cite{Broy:2006:CAS:1134285.1134292,Patrizio2016AAF_Chalmers,qualman2009socialnomics,JansenTale2009}. The increasing amount of stakeholders involved in the software development projects imposes additional challenges to the architecture teams, as the development and design literally cannot be controlled, or even understood, in detail by a single group any more \cite{Patrizio2016AAF_Chalmers}. 

\pat{Add a figure, inspired to the one in ESEM to show the complexity. Move the focus from tool chain but show the complexity, the V-Model style, where in the software life cycle suppliers are involved and the big bang integration.}

The stakeholders in the automotive ecosystem are classified as Original Equipment Manufacturers (OEMs) and its suppliers (Tier-1 and Tier-2). We understand the OEM as the coordinator and platform owner in the automotive ecosystem \cite{KS15,Patrizio2016AAF_Chalmers}. In this paper, this is the VCC. Whereas Tier-1 suppliers are considered direct suppliers to OEM and Tier-2 provide directly to Tier-1 and vice versa, hence, indirect to the OEM. 

The OEM experiences heavy reliance on external developers and subcontractors that complicates coordination through process. Expensive communication and coordination delays during integration are results of outsourcing significant parts of development to suppliers. This form of exponentially growing feature content severely complicates ?big-bang? integration \cite{Eklund2012}. The development is inevitably parallelized; this obviously also holds for the large amounts of externally developed software, which is integrated as black box functionality \cite{Patrizio2016AAF_Chalmers,Broy2009AAF_TUM,Broy:2006:CAS:1134285.1134292}. 

The collaboration between these stakeholders needs to improve the software quality and provide faster, cheaper, and more predictable development \cite{herbsleb2016IntelligentTransparent}. This implies that the automotive (software) industry must identify how this new scenario can be supported at best when an ecosystem of organizations collaborates.

These considerations motivate the need of considering different viewpoints and views from the perspective of specific system concerns, which are relevant for one or more stakeholders. Because of this, it is necessary for members of this ecosystem to agree on a common way of structuring in order to increase overall efficiency within the ecosystem \cite{Patrizio2016AAF_Chalmers,Broy2009AAF_TUM,Broy:2006:CAS:1134285.1134292}. An essential technical problem to solve for this vision is the establishment of standards for interoperability among IPs, both software and hardware, and tools \cite{Broy:2006:CAS:1134285.1134292}. ``Establishing and evolving ecosystems of different partner types might ultimately decide which companies win a market." \cite{Bosch2016Ecosystem}. First attempts, such as AUTOSAR \cite{acm2008autosar} and the Automotive Architecture Framework (AAF) \cite{Patrizio2016AAF_Chalmers,Broy2009AAF_TUM} are being developed. However, researchers and practitioners both identified the need for further research on this emerging type of software ecosystems.

