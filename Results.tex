\section{Results}\label{sec:results}

\subsection{Proposition 5: A more open transparency policy improves the quality of the project and its results.}

This proposition is developed to investigate whether the quality of the project results benefit from a more open transparency policy across companies. During the interview survey, the interviewees were asked a question about the effects of this policy on the project and its results. 

Findings emerging from the interviews are:

{\bf High level of Quality.} The interviewees were unanimous about the positive effects on the quality of the project and its results. The quality improvements are already visible in the early stage of the project and they are confident about the improvements in the long term. An open transparency policy is positive for quality control because of mutual understanding of the project status and as a result, gaining efficiency. Delphi experiences healthy pressure on their activities and forces them to a certain quality level, because the customer is involved in the process of the product. To preserve the quality of the progress, Volvo is not heavily involved in the process.

Accordingly, we elicit the following possible answer to proposition 5:
The findings support this proposition and prove the quality of the project and its results improve due to a more open transparency policy across companies.

\subsection{Proposition 6: Strict contract-based collaboration is an impediment for cross-organisational Continuous Integration and Deployment}

This proposition challenges how practitioners experience a strict (or closed) contract in a cross-organisational setting where companies work together in software engineering projects using Continuous Integration and Deployment (CI\&D). During the interview survey, the interviewees were asked questions on the role of the contract when looking at information sharing and cross-organisational CI\&D. During the research it became clear that information sharing is seen as a crucial factor for cross-organisational collaboration. A closed contract regulates traditional project setups where the customer defines a list of requirements and the supplier has to fulfil it within a given time frame and budget. Originally the automotive industry is traditional and relatively closed. It however emerges that it changing towards greater cross-organisational transparency, participation in open source projects, and becoming a software-intensive sector. While still ongoing, this transition is confirmed by all interviewed stakeholders, and could lead to a cultural change toward adopting more open or agile contracts for cross-organisational CI\&D.


\begin{figure}[htb]
\centering
\includegraphics[width=\columnwidth]{figure/ss_CodeGroup6.png}
\caption{Cultural change toward adopting more open or agile contracts for cross-organisational CI\&D}
\label{fig:towardsAgile}
\end{figure}

Findings emerging from the interviews are:

{\bf Agile contracts favour cross-organisational collaboration.} The interviewees share the opinion that a more open (or agile) contract would be healthier for the project and benefit cross-organisational collaboration. Although some of the project members from both customer and supplier companies are not fully aware of the contract details, they do share the feeling of being restricted. They believe that strict contracts conflict with an agile way of working instead of supporting it, and suggest to adopt more agile contracts, instead, also referred to as Time and Materials (T\&M) contracts\footnote{According to a T\&M contract, the contractor is being billed per hour regardless of the software project duration. If any additional features have to be developed the supplier charges just for the time spent by its employees working on a certain set of tasks [en.wikipedia.org]. This brings high flexibility to accommodate projects with evolving requirements, but also high uncertainty about the related costs.}. 
The interviewees agree that a T\&M contract allows for a better adaptation to project changes, distribution of resources, and it creates shared ownership otherwise hindered by closed contracts. They also argue in favour of a combination of a fixed price and T\&M contract, where stakeholders would agree on the product and cost estimation, but maintain high flexibility on how to produce it. This combination fulfils the need for flexibility and agility, but also the security for the customer. All interviewees made it clear that good collaboration between their companies is important from a legal and contractual perspective to support cross-organisational CI\&D.

{\bf Closed contracts ease negotiation.} For a customer it is (still) more comfortable to work with closed contracts because one has more leverage and binds the supplier to pre-defined deliverables and deadlines. A Volvo manager involved in a RFQ projects, further states that it is hard for suppliers to negotiate with a T\&M or other agile contracts, and that closed contracts make it easier competing with other suppliers.


Accordingly, we elicit the following possible answers to proposition 6:

\begin{enumerate}
\item Cross-organisational CI\&D benefits from a more open collaboration among companies, such as information sharing and adaptation to project changes. A strict contract is an impediment for this way of working related to cross-organisational CI\&D.
\item A strict contract is an impediment for cross-organisational CI\&D. Although there is no direct connection between a strict contract and cross-organisational CI\&D, the way of working related to this method benefits from an open or agile contract.
\item Cross-organisational CI\&D is possible with a strict contract, but synergy effects, i.e. collaboration and flexibility, are in effect when supported by an agile contract. By taking the synergy effects into account, a strict contract can be an impediment for Cross-Organisational CI\&D. An agile contract is difficult for a RFQ due to the open and uncertain characteristics of the contract.
\end{enumerate}


\subsection{Proposition 8: Standards and processes, based on industry-wide data and process standards benefit cross-organisational CI\&D.}

This proposition challenges the interviewees to experience the effects of industry-wide standards and processes in a cross-organisational setting where companies work together in software engineering projects using Continuous Integration and Deployment. The interviewees were asked if they use industry-wise standards and whether they find them beneficial for information sharing, which is important for cross-organisational CI\&D.

Findings emerging from the interviews are:



Accordingly, we elicit the following possible answer to proposition 8:




% {\bf Industry Perspective.} 
% To preserve an open approach to the project, a product manager at a software development company suggests a combination of an agile and fixed price contract by creating project branches to avoid overhead in the main project.
%
%(Y)
% {\bf Maturity.} The automotive software ecosystem needs to adapt to the needs from the stakeholders. The industry is not mature enough, but is improving to adopt cross-organisational Continuous Integration and Deployment.
% Yes: Johnny Karlsson, Darrel Cullen, Lars Mattson, Petter Molder, Jacob Juul, Matti Larborn, 
% No: Anders Lindbom, Michael Svenstam, Mattias Almljum