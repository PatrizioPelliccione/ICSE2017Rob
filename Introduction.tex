\section{Introduction}\label{sec:intro}


Nowadays the automotive business is rapidly and continuously changing, driven by needs of 
electrical, autonomously driving, and connected cars and by the appearance of new major players in
the field, like Google~\cite{Google}, Apple~\cite{Apple,Apple2}, Tesla, UBER, etc.
OEMs are more and more becoming software companies 
and have an increasing and strong motivation
to embrace organization-wide and also cross-organizations Continuous Integration \& Deployment (CI\&D) to yield improvements
in flexibility and cycle time, despite the challenges.
Historically, software was introduced in cars
to optimize the control of the engines. 
Today, according to industry experts, 80\%
to 90\% of the innovation within the automotive industry is
based on electronics and a big part of electronics is software. 

The complexity of the automotive domain naturally leads (driven by the V-model development method) on having different level of abstractions and on splitting the organization in sections, each of them taking care of a specific level, e.g. top level requirements,
architecture, design and software development, hardware, mechanical parts, etc. 
Moreover, it is very impractical to produce all parts comprising a vehicle in-house. Normally OEMs 
ask suppliers to produce parts of the vehicle, and the number of suppliers involved in parallel development can easily
reach a count of 50 different suppliers. 
This means that the knowledge and competence of building a car is spread around the ecosystem composed of several different companies, with different organizations, objectives, competences, constraints, cultures, languages, and geographical placements. 
Therefore, reducing time-to-marked and improving flexibility in the automotive domain necessarily need to involve different actors in the automotive ecosystems.

Currently, the agreements between these companies are mainly based on closed contracts. 
However, our hypothesis, which is beyond the motivation of this paper, is that {\bf closed contracts are a killer for cross-organizations CI\&D}.
We believe that a new way of working between a OEM and its
suppliers is needed in order to seriously adopt organization-wide CI\&D in the automotive domain. This is because only a part of the software is produced in house.
More specifically, the relation between the OEM and its suppliers should change from a contract-based relationship to a more
collaborative one where the OEM and suppliers are part
of the same ecosystem and collaborate for mutual benefit.
More transparency and less contract-based working will then
be some of the key enablers for organization-wide CI\&D.

In order to confirm or reject this hypothesis we formulated the following research question and sub-research questions:

\begin{itemize}
\item {\bf RQ1}: What are the challenges, impediments, and opportunities of applying continuous
integration, delivery, and deployment across the OEM and its suppliers in the
automotive software ecosystem?
\begin{itemize}
\item {\em RQ1a: What is the impact on business model strategies?}
\item {\em RQ1b: What are the risks and the organisational and process impact of increasing
inter-organisational transparency on the relationships?}
\item {\em RQ1c: Which existing concepts can benefit inter-organisational transparency?}
\end{itemize}
\end{itemize}


\pat{Mention the project of Delphi and Volvo Cars as an interesting case to investigate}

The paper is organized as follows: Section~\ref{sec:context} defines the context of the study. Section~\ref{sec:researchMethod} describes the research methodology we followed in order to give an answer to the research questions. Section~\ref{sec:results} presents the results of the study. Section~\ref{sec:discussion} discussed the main findings. Section~\ref{sec:relatedWorks} presents related works, and finally the paper concludes in Section~\ref{sec:conclusion} with final remarks and direction for future works.
