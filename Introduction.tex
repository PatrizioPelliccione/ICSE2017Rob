\section{Introduction}\label{sec:intro}


Nowadays the automotive industry is rapidly and continuously changing, driven by needs of 
electric/hybrid cars, autonomous driving, and connected cars. At the same time, new major players are emerging in the field, like Google~\cite{Google}, Apple~\cite{Apple,Apple2}, Tesla, and UBER, while Original Equipment Manufacturers (OEMs) are incresingly turning into software companies. 
Historically, software was introduced in cars to optimize the control of the engine. 
Today, according to industry experts\footnote{\url{http://www.electronics-eetimes.com/news/innovation-car-90-comes-electronics-and-software}}, 80\% to 90\% of the innovation in the automotive industry is based on electronics, a big part of which is software~\cite{ESEM2016}. 

The V-model\footnote{Originally developed for defense applications by the German DoD - \url{http://www.v-modell-xt.de/}} is the de-facto development method in the automotive domain as also implicitly suggested by the ISO26262 functional safety standard~\cite{iso26262}. The consequence of that is on having 
%The complexity of the automotive domain naturally leads (driven by the V-model development method on having 
different levels of abstractions and on splitting the organization in sections, each of them taking care of a specific level, e.g. top level requirements,
architecture, design and software development, hardware, mechanical parts, etc. 
%\todo{the role of the V-model is unclear. If it is generally adopted by automotive industry, we could rephrase in (1) the V model is the standard de facto. (2) due to the complexity of the domain, the V-model causes a silos effect ... This causes separations that hinders infor sharing and synergy.})
This can cause a silos effect and separations that hinder information sharing and synergy.

Moreover, it is very impractical to produce in-house all parts comprising a vehicle. Normally, OEMs 
ask suppliers to produce parts of the vehicle, and the number of suppliers involved in parallel development can easily reach a count of 50 different suppliers. 
This means that the knowledge and competence of building a car is spread across an ecosystem composed of several different companies, each with potentially different internal organizations, objectives, competences, constraints, cultures, languages, and geographical locations. 

Nowadays OEMs have an increasing and strong motivation
to embrace system-wide Continuous Integration \& Deployment (CI\&D). The goal is to improve
in flexibility and cycle time, despite the challenges mentioned above. 
%Therefore, reducing time-to-marked and improving flexibility in the automotive domain necessarily need to involve different actors in the automotive ecosystems.
Since automotive development relies on complex supply chains, the agility across company boundaries becomes a crucial concern.

In this paper we want to investigate %\todo{what we claim here is not fully aligned with the RQs} 
what are the challenges and impediments of system-wide CI\&D in the automotive ecosystem. 
%scaling software engineering agility \todo{added SE: to make the SE relevance more evident} beyond the company boundaries.  
%However, our hypothesis, which is beyond the motivation of this paper, is that {\bf closed contracts are a killer for cross-organizations CI\&D}.
Specifically, we focus on {\em contracts} that regulate the agreements between these companies 
%are mainly based on closed contracts.
and {\em transparency} intended as the degree/level of information that is shared between the various companies in the ecosystem. 

Focusing on the engineering of the software of modern and near future vehicles, our study aims at providing an answer to the following research questions:

%We believe that a new way of working between a OEM and its
%suppliers is needed in order to seriously adopt organization-wide CI\&D in the automotive domain. This is because only a part of the software is produced in house.
%More specifically, the relation between the OEM and its suppliers should change from a contract-based relationship to a more
%collaborative one where the OEM and suppliers are part
%of the same ecosystem and collaborate for mutual benefit.
%More transparency and less (closed-)contract-based working will then
%be some of the key enablers for organization-wide CI\&D.
%
%In order to confirm or reject this hypothesis we formulated the following research questions: % and sub-research questions:

%\begin{itemize}
%\item {\bf RQ1}: What are the challenges, impediments, and opportunities of applying continuous
%integration, delivery, and deployment across the OEM and its suppliers in the
%automotive software ecosystem?
\begin{itemize}
%\item {\em RQ1: What is the impact on business model strategies?}
\item {\em RQ1: What are the risks and/or benefits of increasing inter-organisational transparency?}
\item {\em RQ2: Is there a lack or overload of information that is exchanged across organizations?} 
\item {\em RQ3: Are closed-contracts an impediment for scaling agility across company boundaries, and 
are industry-wide standards and processes shared among organizations an enabler?} 
%what are the challenges of changing this collaboration model?} %What are the risksbenefits of increasing inter-organisations transparency?} %ch existing concepts can benefit inter-organisational transparency?}
%\item {\em RQ3: What are the main challenges while improving inter-organisations transparency?}
%\item {\em RQ3: What can be the impacts on business model strategies?}
\end{itemize}
%\end{itemize}

To give an answer to these research questions we performed semi-structured interviews within Volvo Cars and 
%Delphi 
one of their larger suppliers in the context of a pilot project within Volvo Cars. This pilot project is experimenting a more open and transparent way of working between these two companies. Therefore this pilot project is an ideal setting for investigating the effects of changing the way of collaborating between OEMs and suppliers.  
The pilot project is a large and complex project with over 100 engineers, and it will be closer to 200 if we include also developers. 
% and engineers if one count all types . 
%There are close to 30 different SCRUM and Agile teams. %It is classified\todo{By whom? What does this mean?} as the highest degree of complexity that can be done between the companies. 
In an attempt to reduce the complexity, developers from both companies work in the same office space, and they are organized in around 30 different agile teams. %\todo{scrum is (!) agile: is the difference relevant?}

The main findings of the study are the following:
\begin{itemize}
\item {\em RQ1} - Inter-organisational transparency is not a necessary condition for inter-organizational CI\&D. However, transparency is considered positive for various reasons. %and creates positive synergy effects in terms of efficiency, trust, and mutual understanding while avoiding useless stressful situations. Transparency is also considered positive in terms of increasing the overall the project quality. 
There exist strategies to facilitate sharing inter organizations, however, the automotive industry experiences difficulties to share information, and manage responsibilities and IPR.
\item {\em RQ2} - Transparency has good effects on information sharing among the members of the project (which belong to different companies). However transparency does not help by ``per se" to provide a holistic project overview.
\item {\em RQ3} - Closed-contracts are an impediment for inter-organisational CI\&D; however, closed-contracts facilitate negotiations between different organizations. Industry-wide standards and processes and open source initiatives are seen as positive for promoting collaborations, knowledge sharing, and communication.
\end{itemize}
%The findings of the paper are ...\pat{findings}

%The biggest challenges that emerged are: 
%
%\begin{itemize}
%\item  The automotive industry experiences difficulties to share information in the ecosystem, as well as to manage responsibilities and Intellectual Property Rights (IPRs). 
%\item When the collaboration between different organizations is regulated though more ``open-contracts", it is not obvious  how to manage negotiations and responsibility sharing.
%\item Means and strategies to share a ``big picture" of the project among the different stakeholders should be identified. A holistic view of the project could be beneficial for all stakeholders and can increase project efficiency and quality. 
%\end{itemize}

%\pat{Rogardt: is the above enough or should I say something more}

%\pat{Rogardt might help here. I don't know much about this project and also, how much we can say?}

The paper is organized as follows: Section~\ref{sec:context} defines the context of the study. Section~\ref{sec:researchMethod} describes the research methodology we followed in order to give an answer to the research questions. Section~\ref{sec:results} presents the results of the study. Section~\ref{sec:discussion} discussed the main findings. Section~\ref{sec:relatedWorks} presents related works, and finally the paper concludes in Section~\ref{sec:conclusion} with final remarks and direction for future works.