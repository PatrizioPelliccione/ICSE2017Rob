\section{Discussion}\label{sec:discussion}

%\pat{How much upfront? What are the key artifacts that should be maintained?}


\subsection{Industry Perspective}

This study investigated a pilot project of Volvo Cars and one of its suppliers to investigate innovative ways of working and innovative collaboration models of the two organizations. Industries that are thinking or that are already in the process of opening their platforms towards an ecosystem perspective, or that simply are joining ecosystems might benefit from our findings.

%The proposition challenges interviewees to critically evaluate the impact of cross-organisational transparency on the business model strategy of their organisation. They were asked about the impact on the business model strategy on a functional and organisational level. 
%In general, the interviewees do not have an overview of the business model strategies of their organisation. 
%In addition to the findings already discussed above, in this section we want 
It is important to highlight also that system-wide CI\&D could have an impact on both organizational and functional levels. 
%had a bigger impact in the early stage of the pilot project, but is getting less now the project is ongoing. 
%
%However, from our study it emerges that system-wide CI\&D %the interview results support the proposition, and cross-organisational transparency and CI\&D 
%could have an impact on business model strategies. %This answer to proposition 9 is deduced from the following findings:
%
%\noindent {\bf Impact on organisational level.} The organisational 
The impact on organisational level is perceived differently by the two participating companies. For Volvo Cars, it is perceived as one of the many projects and since they do not have the integration and end-responsibility for the product, it has less of an impact. However, for the supplier company, it is a high-risk experiment due to the complexity of the project. The employees of the supplier confirm that it has the highest complexity level the company accepted in its history.

%Th impact on functional level.} On functional level, 
The interviewees of both companies agree that the impacts \todo{impact?} on functional level of system-wide CI\&D %effects derived from the impact of cross-organisational transparency and CI\&D 
can be quantified in gain-in-efficiency, quality, and time-to-market. Also, CI\&D has less side effects than intermittent integration, in case of big-bang integration.  

Finally, the interviewees state that, to date, system-wide CI\&D is not mature enough to see or foresee a big impact on business model strategies. 

\subsection{Academic Perspective}

Researchers that are interested in scaling agile methodologies to the entire organization or to inter-organizational collaborations might find in this paper interesting findings coming from an industrial project in a challenging domain that is witnessing a profound transformation in the last years. 

Researchers working on ecosystems can find in this paper an interesting example of ecosystem in the challenging automotive domain. Innovative ways to share information and knowledge within the ecosystem are needed. %We also considered only one OEM and one supplier. 
We also expect that different levels of sharing will be required within the same ecosystem, according to degree of closeness to the OEM and the level of trust, and according to the purpose of the collaboration.

For what concerns contracts, our findings say that traditional contracts are an impediment to system-wide CI\&D. However these types of contracts work well for what concerns negotiation and IPR. Innovative contracts and/or ways of working are needed.

% {\bf Industry Perspective.} 
%  To preserve an open approach to the project, a product manager at a software development company suggests a combination of an agile and fixed price contract by creating project branches to avoid overhead in the main project.
% 
%  {\bf Maturity.} The automotive software ecosystem needs to adapt to the needs from the stakeholders. The industry is not mature enough, but is improving to adopt cross-organisational Continuous Integration and Deployment.
% % Yes: Johnny Karlsson, Darrel Cullen, Lars Mattson, Petter Molder, Jacob Juul, Matti Larborn, 
% % No: Anders Lindbom, Michael Svenstam, Mattias Almljum

%\subsection{Open source}




%\rog{

%Discussion:
%
%One thing that surprised us analyzing the interviews, was the need for open source to obtain both more transparency and get CI\&D to run more smoothly. A Volvo employee said: ?I believe CI\&D is a model we must strive. I mean if you going to developing a very large system based on open standards. I think having this transparency should come natural, otherwise we will not. Volvo could order a Linux system and get this in a traditional way. We get binary report and get the delivery at the end and we could be happy. We will lose something, because we are building this software together and we have the rights to won and use them the future. Having Transparency and CI really gives a benefit to Volvo in the future reuse this assets, knowledge and experience.?. Another from Volvo is agree, but point out some problems as well, ?Yes. Open source is a very strong area related to CI as you say. Like you said, the tools are open source and the automotive industry is trying to keep this an open-source as possible. There are some week areas, like code coverage tools. They are very week, unfortunately, so we have to use commercial tools for that.?  Also the suppliers try to be as open as possible: ?? Some sources necessarily close. But we are trying to be as open as possible as requested.?.  It is also important to be active participation in the developing for open source: ?We are trying to be a good open source member by also upstreaming strategies, especially around GENIVI and Yocto. Here we find issues or improvements in the tools and we try to pass them back what we can. That is one of the strategies of the Supplier to work more open towards the open-source community as a give and receiver. We are working on that.? But, this might even be easier for Volvo: ?It is harder for the supplier. Since they in the end have to deliver the product on time and working after stream at least in the beginning it takes more time, because you need to align with the parties outside project. It might be a little bit problematic form them I think.?
%
%One of the strongest point is regarding knowledge of the tool, ?Let?s say for example we are using proprietary technology. Are the Supplier or Volvo proprietary, so that means that one of the parties have to adjust to this technology. The other part is always going to be behind and have difficulty to understand how everything should work.?
%
%}