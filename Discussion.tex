\section{Discussion}\label{sec:discussion}

%\pat{How much upfront? What are the key artifacts that should be maintained?}


\subsection{Industry Perspective}

This study investigated a pilot project of Volvo Cars and one of its suppliers to investigate innovative ways of working and innovative collaboration models of the two organizations. Industries that are thinking or that are already in the process of opening their platforms towards an ecosystem perspective, or that simply are joining ecosystems might benefit from our findings.

%The proposition challenges interviewees to critically evaluate the impact of cross-organisational transparency on the business model strategy of their organisation. They were asked about the impact on the business model strategy on a functional and organisational level. 
%In general, the interviewees do not have an overview of the business model strategies of their organisation. 
In addition to the findings already discussed above, in this section we want to highlight that system-wide CI\&D could have an impact on both organizational and functional levels. 
%had a bigger impact in the early stage of the pilot project, but is getting less now the project is ongoing. 

%However, from our study it emerges that system-wide CI\&D %the interview results support the proposition, and cross-organisational transparency and CI\&D 
%could have an impact on business model strategies. %This answer to proposition 9 is deduced from the following findings:
%
%\noindent {\bf Impact on organisational level.} The organisational 
The impact on organisational level is perceived differently by the two companies companies. For Volvo Cars, it is perceived as one of the many projects and since they do not have the integration and end-responsibility for the product, it has less of an impact. However, for the supplier company, it is a high-risk experiment due to the complexity of the project. The employees of the supplier confirm that it has the highest complexity level the company accepted in its history.

%Th impact on functional level.} On functional level, 
The interviewees of both companies agree that the impacts on functional level of system-wide CI\&D %effects derived from the impact of cross-organisational transparency and CI\&D 
can be quantified in gain in efficiency, quality and time-to-market. Also CI\&D has less side effects than intermittent integration, in case of big bang integration.  

Finally, the interviewees state that system-wide CI\&D is not mature enough to see or foresees a big impact on business model strategies. 

\subsection{Academic Perspective}

Researchers that are interested in scaling agile methodologies to the entire organization or to inter-organizational collaborations might find in this paper interesting findings coming from an industrial project in a challenging domain that is living a profound transformation in these years. 

Researchers working on ecosystems can find in this paper an interesting example of ecosystem in the challenging automotive domain. Innovative ways to share information and knowledge within the ecosystem are needed. We also considered only one OEM and one supplier. We expect that different levels of sharing will be required according to degree of closeness to the OEM and the level of trust, and according on the purpose of the collaboration.

For what concerns contracts, our findings say that closed-contracts are an impediment to system-wide CI\&D, however they work well for what concern negotiation and IPR. Innovative contracts and/ot way of working are needed.

% {\bf Industry Perspective.} 
%  To preserve an open approach to the project, a product manager at a software development company suggests a combination of an agile and fixed price contract by creating project branches to avoid overhead in the main project.
% 
%  {\bf Maturity.} The automotive software ecosystem needs to adapt to the needs from the stakeholders. The industry is not mature enough, but is improving to adopt cross-organisational Continuous Integration and Deployment.
% % Yes: Johnny Karlsson, Darrel Cullen, Lars Mattson, Petter Molder, Jacob Juul, Matti Larborn, 
% % No: Anders Lindbom, Michael Svenstam, Mattias Almljum