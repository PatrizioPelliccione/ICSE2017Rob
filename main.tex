% This is "sig-alternate.tex" V2.0 May 2012
% This file should be compiled with V2.5 of "sig-alternate.cls" May 2012
%
% This example file demonstrates the use of the 'sig-alternate.cls'
% V2.5 LaTeX2e document class file. It is for those submitting
% articles to ACM Conference Proceedings WHO DO NOT WISH TO
% STRICTLY ADHERE TO THE SIGS (PUBS-BOARD-ENDORSED) STYLE.
% The 'sig-alternate.cls' file will produce a similar-looking,
% albeit, 'tighter' paper resulting in, invariably, fewer pages.
%
% ----------------------------------------------------------------------------------------------------------------
% This .tex file (and associated .cls V2.5) produces:
%       1) The Permission Statement
%       2) The Conference (location) Info information
%       3) The Copyright Line with ACM data
%       4) NO page numbers
%
% as against the acm_proc_article-sp.cls file which
% DOES NOT produce 1) thru' 3) above.
%
% Using 'sig-alternate.cls' you have control, however, from within
% the source .tex file, over both the CopyrightYear
% (defaulted to 200X) and the ACM Copyright Data
% (defaulted to X-XXXXX-XX-X/XX/XX).
% e.g.
% \CopyrightYear{2007} will cause 2007 to appear in the copyright line.
% \crdata{0-12345-67-8/90/12} will cause 0-12345-67-8/90/12 to appear in the copyright line.
%
% ---------------------------------------------------------------------------------------------------------------
% This .tex source is an example which *does* use
% the .bib file (from which the .bbl file % is produced).
% REMEMBER HOWEVER: After having produced the .bbl file,
% and prior to final submission, you *NEED* to 'insert'
% your .bbl file into your source .tex file so as to provide
% ONE 'self-contained' source file.
%
% ================= IF YOU HAVE QUESTIONS =======================
% Questions regarding the SIGS styles, SIGS policies and
% procedures, Conferences etc. should be sent to
% Adrienne Griscti (griscti@acm.org)
%
% Technical questions _only_ to
% Gerald Murray (murray@hq.acm.org)
% ===============================================================
%
% For tracking purposes - this is V2.0 - May 2012

%\documentclass[10pt, conference, compsocconf]{IEEEtran}
 \documentclass[10pt,conference]{IEEEtran} 
%\documentclass{sig-alternate}
\usepackage{url}
\usepackage{balance}
\usepackage{hyperref}
\usepackage{graphicx}
\usepackage{xspace}
\usepackage{color}
\usepackage{pifont}
\usepackage{xcolor,colortbl}

\usepackage[framemethod=TikZ]{mdframed}
\usepackage{lipsum}
\mdfdefinestyle{ExampleFrame}{
	linecolor=black,
	linewidth=1pt,
	frametitlerule=true,
	frametitlebackgroundcolor=gray!20,
	innertopmargin=\topskip,
	roundcorner=5pt
}


\newenvironment{widequotation}{\list{}{\listparindent 1.5em \itemindent\listparindent
		\rightmargin 0pt \parsep 0pt plus 1pt}\item\relax}
{\endlist}
\def\signed#1{{\leavevmode\unskip\nobreak\hfil\penalty50\hskip2em
		\hbox{}\nobreak\hfil\raise-1pt\hbox{#1}%
		\parfillskip=0pt \finalhyphendemerits=0 \endgraf}}
\newsavebox\mybox
\newenvironment{aquote}[1]
	{\savebox\mybox{(#1)}\begin{widequotation}\itshape``\ignorespaces}
	{\unskip"\signed{\usebox\mybox}\end{widequotation}}


\sloppy

\newcommand {\pat}[1]{[{\bf \underline{Patrizio}}: {\bf #1}]}
\newcommand {\rog}[1]{[{\bf \underline{Rogardt}}: {\bf #1}]}
\newcommand{\todo}[1]{\textcolor{blue}{\ding{46}~{\sf todo}~#1}}

%\newcommand{\definition}[2]{\noindent \textbf{\emph{Definition #1}} (#2)}
\newcommand{\ttransition}[2]{\stackrel{#1}{\longrightarrow^{#2}}}
\newcommand{\ntransition}[1]{\longrightarrow^{#1}}
\newcommand{\transition}[1]{\stackrel{#1}{\rightarrow}}
\newcommand{\Transition}[1]{\stackrel{#1}{\Rightarrow}}
\newcommand{\freccia}[1]{\mathop{\stackrel{#1} {\longrightarrow}} }
\newcommand{\ug}[1]{\mathop{=}\limits^{#1}_{}}
\newcommand{\barra}[1]{\overline{#1}}
\newcommand{\eqdef}{\stackrel{def}{=}}


\newcommand{\footlabel}[2]{%
    \addtocounter{footnote}{1}%
    \footnotetext[\thefootnote]{%
        \addtocounter{footnote}{-1}%
        \refstepcounter{footnote}\label{#1}%
        {\footnotesize #2}%
    }%
    $^{\ref{#1}}$%
}

\newcommand{\footref}[1]{%
    $^{\ref{#1}}$%
}

\usepackage{listings}

% colors
\definecolor{mygreen}{rgb}{0,0.6,0}
\definecolor{mygray}{rgb}{0.5,0.5,0.5}
\definecolor{mymauve}{rgb}{0.58,0,0.82}
\definecolor{light-gray}{gray}{0.85}

% listing settings
\lstset{ %
  backgroundcolor=\color{white},   % choose the background color; you must add \usepackage{color} or \usepackage{xcolor}
  basicstyle=\footnotesize,        % the size of the fonts that are used for the code
  breakatwhitespace=false,         % sets if automatic breaks should only happen at whitespace
  breaklines=true,                 % sets automatic line breaking
  captionpos=b,                    % sets the caption-position to bottom
  commentstyle=\color{mygreen},    % comment style
  deletekeywords={...},            % if you want to delete keywords from the given language
  escapeinside={\%*}{*)},          % if you want to add LaTeX within your code
  extendedchars=true,              % lets you use non-ASCII characters; for 8-bits encodings only, does not work with UTF-8
  frame=single,                    % adds a frame around the code
  keepspaces=true,                 % keeps spaces in text, useful for keeping indentation of code (possibly needs columns=flexible)
  keywordstyle=\color{blue},       % keyword style
  language=C,                      % the language of the code
  morekeywords={*,...},            % if you want to add more keywords to the set
  numbers=left,                    % where to put the line-numbers; possible values are (none, left, right)
  numbersep=5pt,                   % how far the line-numbers are from the code
  numberstyle=\tiny\color{mygray}, % the style that is used for the line-numbers
  rulecolor=\color{black},         % if not set, the frame-color may be changed on line-breaks within not-black text (e.g. comments (green here))
  showspaces=false,                % show spaces everywhere adding particular underscores; it overrides 'showstringspaces'
  showstringspaces=false,          % underline spaces within strings only
  showtabs=false,                  % show tabs within strings adding particular underscores
  stepnumber=1,                    % the step between two line-numbers. If it's 1, each line will be numbered
  stringstyle=\color{mymauve},     % string literal style
  tabsize=2,                       % sets default tabsize to 2 spaces
  caption=A program                % show the filename of files included with \lstinputlisting; also try caption instead of title
}

\begin{document}


\title{CI\&D in the Automotive Ecosystem: Challenges and Impediments}




\author{\IEEEauthorblockN{Rob van der Valk\IEEEauthorrefmark{1}, Patrizio Pelliccione\IEEEauthorrefmark{2}, Patricia Lago\IEEEauthorrefmark{1}, Rogardt Heldal\IEEEauthorrefmark{2}, and Eric Knauss\IEEEauthorrefmark{2}}
\IEEEauthorblockA{\IEEEauthorrefmark{2}Chalmers University of Technology | University of Gothenburg\\
Department of Computer Science and Engineering,
Gothenburg, Sweden\\
Email: patrizio.pelliccione@gu.se, heldal@chalmers.se, eric.knauss@gu.se}
%\and
\IEEEauthorblockA{\IEEEauthorrefmark{1}Computer Science Institute, Vrije Universiteit Amsterdam, The Netherlands\\
Email: robvdvalk@gmail.com and p.lago@vu.nl}
%\IEEEauthorblockA{
%Email: name@xyz.com}
}

% conference papers do not typically use \thanks and this command
% is locked out in conference mode. If really needed, such as for
% the acknowledgment of grants, issue a \IEEEoverridecommandlockouts
% after \documentclass

% for over three affiliations, or if they all won't fit within the width
% of the page, use this alternative format:
% 
%\author{\IEEEauthorblockN{Michael Shell\IEEEauthorrefmark{1},
%Homer Simpson\IEEEauthorrefmark{2},
%James Kirk\IEEEauthorrefmark{3}, 
%Montgomery Scott\IEEEauthorrefmark{3} and
%Eldon Tyrell\IEEEauthorrefmark{4}}
%\IEEEauthorblockA{\IEEEauthorrefmark{1}School of Electrical and Computer Engineering\\
%Georgia Institute of Technology,
%Atlanta, Georgia 30332--0250\\ Email: see http://www.michaelshell.org/contact.html}
%\IEEEauthorblockA{\IEEEauthorrefmark{2}Twentieth Century Fox, Springfield, USA\\
%Email: homer@thesimpsons.com}
%\IEEEauthorblockA{\IEEEauthorrefmark{3}Starfleet Academy, San Francisco, California 96678-2391\\
%Telephone: (800) 555--1212, Fax: (888) 555--1212}
%\IEEEauthorblockA{\IEEEauthorrefmark{4}Tyrell Inc., 123 Replicant Street, Los Angeles, California 90210--4321}}





\maketitle
\begin{abstract}
The automotive domain is rapidly changing because of new and unavoidable challenges, like autonomous vehicles, connected cars and electrification, and of new players that are entering the market, like Tesla, Google, Apple, Uber, etc. 
Most of the innovation in automotive is nowadays coming from electronics and software and as a matter of facts OEMs are becoming software companies. 
The pressure of reducing the time to market and improving flexibility while keeping quality are leading motivations for these companies to embrace system-wide Continuous Integration \& Deployment (CI\&D). 

In this paper we investigate the challenges and impediments of system-wide CI\&D in the automotive ecosystem, i.e. CI\&D that involves both OEM and its software suppliers. In particular we focus on {\em contracts} that regulate the agreements between these companies 
%are mainly based on closed contracts.
and {\em transparency} intended as the degree/level of information that is shared between the various companies in the ecosystem. The results of the study provide useful insights for practitioners that work in a similar ecosystem, and on the other side the identified challenges and impediments define a research agenda for researchers.
\end{abstract}

\begin{IEEEkeywords}
Automotive ecosystem, relationships OEM suppliers, transparency, win-win, new business model, contracts.

\end{IEEEkeywords}

\IEEEpeerreviewmaketitle

\section{Introduction}\label{sec:intro}


Nowadays the automotive business is rapidly and continuously changing, driven by needs of 
electrical, autonomously driving, and connected cars and by the appearance of new major players in
the field, like Google~\cite{Google}, Apple~\cite{Apple,Apple2}, Tesla, UBER, etc.
OEMs are more and more becoming software companies 
and have an increasing and strong motivation
to embrace organization-wide Continuous Integration \& Deployment (CI\&D) to yield improvements
in flexibility and cycle time, despite the challenges.
Historically, software was introduced in cars
to optimize the control of the engines. 
Today, according to industry experts, 80\%
to 90\% of the innovation within the automotive industry is
based on electronics and a big part of electronics is software. 

The complexity of the automotive domain naturally leads (driven by the V-model development method) on having different level of abstractions and on splitting the organization in sections, each of them taking care of a specific level, e.g. top level requirements,
architecture, design and software development, hardware, mechanical parts, etc. 
Moreover, it is very impractical to produce all parts comprising a vehicle in-house. Normally OEMs 
ask suppliers to produce parts of the vehicle, and the number of suppliers involved in parallel development can easily
reach a count of 50 different suppliers. 
This means that the knowledge and competence of building a car is spread around the ecosystem composed of several different companies, with different organizations, objectives, competences, constraints, cultures, languages, and geographical placements. 
The agreements between these companies are currently mainly based on contracts. 

Our hypothesis, which is beyond the motivation of this paper, is that {\bf contracts are a killer for organization-wide CI\&D}.
We believe that a new way of working between a OEM and its
suppliers is needed in order to seriously adopt organization-wide CI\&D in the automotive domain. More specifically, the relation between the OEM and its suppliers should change from a contract-based relationship to a more
collaborative one where the OEM and suppliers are part
of the same ecosystem and collaborate for mutual benefit.
More transparency and less contract-based working will then
be some of the key enablers for organization-wide CI\&D.

In order to confirm or reject this hypothesis we formulated the following research questions:

\begin{itemize}
\item {\bf RQ1:}
\item {\bf RQ2:}
\item ...
\end{itemize}

\pat{Mention the project of Delphi and Volvo Cars as an interesting case to investigate}

The paper is organized as follows: Section~\ref{sec:context} defines the context of the study. Section~\ref{sec:researchMethod} describes the research methodology we followed in order to give an answer to the research questions. Section~\ref{sec:results} presents the results of the study. Section~\ref{sec:discussion} discussed the main findings. Section~\ref{sec:relatedWorks} presents related works, and finally the paper concludes in Section~\ref{sec:conclusion} with final remarks and direction for future works.

\section{Context of the study}\label{sec:context}


This work is made in the context of the {\em Next-Generation Electrical Architecture (NGEA)} and {\em Next-Generation Electrical Architecture - step 2 (NGEAs)} projects, funded by Vinnova~\cite{Vinnova}. 
These projects are coordinated by Volvo Cars and involve the Chalmers University of Technology, some research centers in Sweden and many suppliers of the OEM, including Autoliv, Arccore, Combitech, Cybercom, Knowit, Prevas, \AA F-Technology, Semcom, and Qamcom. The projects aim to develop new software processes and proof of concepts to strengthen the competitiveness of the automotive industry in Sweden. The main objectives of the projects are to investigate (i) the transition of Volvo Cars towards continuous integration and deployment, (ii) new business models and innovative ways of working within the automotive ecosystem, and (iii) vehicles as part of system of systems. 
%Included in this project are sub projects or work packages that focus on 1) management, administration and business intelligence studies, 2) continuous deployment of architectural and development strategic viewpoints, and 3) automobiles as a system within an automotive software ecosystem \cite{Vinnova}. 
In this paper we mostly focus on point (ii) even though there will be some impact on point~(i). Since OEMs are becoming software companies, nowadays the automotive domain is increasingly attracting the attention of the software engineering community.  
%\todo{why relevant for ICSE?}

The automotive ecosystem consists of cross-organizational collaborations among automotive suppliers. It is characterized by relying on complex supplier networks and strong dependence on hardware and software development \cite{Knauss14d.:towards,Broy:2006:CAS:1134285.1134292}.
The current automotive industry is {\em closed}, with strict organizational boundaries, stiff processes, established business models and a straightforward value creation~\cite{ConnectedVehicle2012}.


Nowadays, a vehicle is a {\em driving software package} as compared to the vehicles of not even ten years ago. J\"orgen M\"ossinger, VP for automotive systems integration at Bosch, said: ``Electronics and especially software are the main sources of automotive innovation today."~\cite{Mossinger2010SoftwareAutomotive}. The Boston Consulting Group estimates that the total costs of electronic parts will rise from 20\% of the value in a typical car in 2004 to 40\% in 2015. Software, instead of hardware, has become the differentiating factor% of products
~\cite{ConnectedVehicle2012,hbr2015hardwaresoftware,Mossinger2010SoftwareAutomotive,Broy:2006:CAS:1134285.1134292} %. In the past, hardware was the differentiator 
between companies and their products. This evolution of the automotive industry, illustrated by the exponential increase of software, creates new challenges regarding software integration, development, deployment, and maintenance. Therefore, its development needs to support the related integration and evolution of time~\cite{Broy:2006:CAS:1134285.1134292,qualman2009socialnomics,JansenTale2009}. The increasing amount of stakeholders involved in the software development projects imposes additional challenges to the architecture teams, as the development and design literally cannot be controlled, or even understood, in detail by a single group any more. 

The stakeholders in the automotive ecosystem are classified as OEMs (e.g. Volvo Cars) and its suppliers (Tier-1 and Tier-2). In general the OEM is the coordinator and platform owner in the automotive ecosystem~\cite{KS15}. Tier-1 suppliers are considered direct suppliers to OEM and Tier-2 companies are a second level of suppliers, indirect to the OEM and directly connected to Tier-1. % and vice versa, hence, indirect to the OEM. \todo{is this relevant+understandable by ICSE readers?}

\begin{figure}[htb]
\centering
\includegraphics[width=\columnwidth]{figure/Closed-contract-collaboration.pdf}
\caption{Collaboration based on closed contracts}
\label{fig:closedContractCollaboration}
\end{figure}

Therefore, OEMs experience heavy reliance on external developers and subcontractors; this complicates coordination throughout the entire development process. Expensive communication and coordination delays during integration are results of outsourcing significant parts of development to suppliers. %This form of exponentially growing feature content severely complicates ``big-bang" integration~\cite{Eklund2012}. The development is inevitably parallelized; this obviously also holds for the large amount of externally developed software, which is integrated as black-box functionality~\cite{Patrizio2016AAF_Chalmers,Broy2009AAF_TUM,Broy:2006:CAS:1134285.1134292}. 

As shown in Figure~\ref{fig:closedContractCollaboration} the software engineering process follows the V-model, with the development on the left-hand side and verification on the right-hand side. The development at the level of components is parallelized among the different suppliers, and internal in-house development. The degree of parallelism can easily reach level 50 (i.e. 50 parallel developments). Once the collaboration between the OEM and its suppliers is regulated by closed-contract, the parallel developments represented in Figure~\ref{fig:closedContractCollaboration} start by signing a contract and then after months the large amount of externally developed software come back to the OEM and it is integrated as black-box functionality~\cite{Broy2009AAF_TUM,Broy:2006:CAS:1134285.1134292}. 
%the produced ECUs will be provided back to the OEM after some months with few communication during the production period.
This leads to the above mentioned ``big bang" integration~\cite{Eklund2012}. %: the developed Electronic Control Units (ECUs) (which include both hardware and software) come back to the OEM and integration starts. 
It is at this stage that many misunderstandings, conflicting interpretations, wrong assumptions, etc. are discovered.
It is easy to understand that closed-contract relations between the OEM and the suppliers can slow down the system-wide CI\&D.  %\todo{this last statement is disconnected from the rest of the story: miss the link between the mentioned integration problems and closed contracts}

%The collaboration between the stakeholders in the ecosystem needs to improve the software quality and provide faster, cheaper, and more predictable development~\cite{herbsleb2016IntelligentTransparent}. 
%
%This implies that the automotive (software) industry must identify how this new scenario\todo{what is the new scenario?} can be supported at best when an ecosystem of organizations collaborates.
%
%\todo{there is a gap between these two paragraphs: collaboration should be supported by new tools, like views and associated viewpoints to communicate the right architectural knowledge, etc.}
%
%These considerations motivate the need of considering new types of viewpoints and views\todo{add citation} from the perspective of specific system concerns, which are relevant for one or more stakeholders collaborating in the automotive ecosystem. Because of this, it is necessary for members of this ecosystem to agree on a common way of structuring\todo{what do you mean? structuring what?} in order to increase overall efficiency within the ecosystem \cite{Patrizio2016AAF_Chalmers,Broy2009AAF_TUM,Broy:2006:CAS:1134285.1134292}. An essential technical problem to solve for this vision is the establishment of standards for interoperability among IPs, both software and hardware, and tools \cite{Broy:2006:CAS:1134285.1134292}. ``Establishing and evolving ecosystems of different partner types might ultimately decide which companies win a market." \cite{Bosch2016Ecosystem}. First attempts, such as AUTOSAR \cite{acm2008autosar} and the Automotive Architecture Framework (AAF) \cite{Patrizio2016AAF_Chalmers,Broy2009AAF_TUM} are being developed. However, researchers and practitioners both identified the need for further research on this emerging type of software ecosystems.
%

\section{Research Method}\label{sec:researchMethod}

This section describes the research methodology used to address the research questions.
Our data collection is based on semi-structured interviews. 
The interview protocol follows the seven-stages approach by Kvale \& Brinkmann~\cite{Kvale2015Interviews}, which covers preparation, execution and reporting. The seven stages are: Thematising, Designing, Interviewing, Transcribing, Analysing, Verifying, and Reporting. The way we used them in our study is described in the rest of this section. 

The interview results aim at providing in-depth analysis on the current situation, as well as explore the desired (future) situation. %The results also provide the foundation for the Reference Model\todo{What is that?}~\cite{Blessing:2009:DDR:1717807}. %The goal of this section is to provide a clear description of the interview process applied in this research to support and clarify the results and present the results.

\subsection{Seven stages}

\noindent {\bf Thematising}: This phase decides on the purpose and subject matter, as well as the methods to be applied. The purpose of the interview survey is to require in-depth information on the topics of CI\&D %Continuous Integration \& Delivery 
and Ecosystem \& Transparency in the automotive industry. These topics are discussed with representatives of Volvo Cars and its suppliers (Tier-1 and Tier-2). % Also consultancy companies who work closely with the automotive industry are interviewed for this project.\todo{what do you mean "for this project"? delete?}

\noindent {\bf Designing}: The seven stages of the approach are designed for the study. The structure of the interview survey, transcribing details, analysis protocol \cite{saldana2015coding}, verification and reporting are discussed in detail. Beforehand, the interviewee is informed on the interview details, topics and requested permission for the purpose of recording the conversation for the transcript. %By doing this beforehand, 
%the interviewee is familiar with the topics and allows him or her to reflect on the research topics upfront.

\noindent {\bf Interviewing}: The interview has been structured as a semi-structured interview where open questions guide the interview process towards a list of goals that need to be achieved in the interview. Semi-structured interviews are very suitable for exploratory research \cite{Runeson:2009:GCR:1519313.1519324}. The interview is organized in three categories of questions: introduction, main questions and cool-off.
The {\em introduction} is meant to create context / background, and to provide and maintain an informal and interactive atmosphere. %For the purpose of the research, it was necessary that the interviewee is able to explain as much as possible about her or her knowledge and experience in the own field of expertise in a limited time span. 
The {\em main part} of questionnaire is aimed at the actual research. The questions aim to guide the interview, but be dynamic and interactive at the same time. This dynamic and interactive nature of interviewing is useful for exploration and development of interesting and unexpected ideas brought up by the interviewee, which are less possible with other (more structured) methodologies, such as questionnaire surveys.
At the end of the interview, or {\em cool-off}, the interviewee is given time to ask some questions about the topics, this could be topic or themes that are not mentioned in the interview or could be important for the research. \pat{Add a link to the intervieew questions} We selected the candidates to be interviewed within the pilot project by trying to cover different expertise and by selecting people that are knowledgeable about the topic.
An overview of the selected interviewees might be found in Table~\ref{tab:population}; the years in brackets refer to industry experience, without brackets refer to experience in that specific role.

 \begin{table}[htb]
 \centering
 \begin{tabular}{|c|c|p{3.4cm}|c|}\hline
 {\bf No.} & {\bf Company} & {\bf Role within company} & {\bf Exp. in years}\\ \hline
 1 & Volvo Cars & Project Manager RFQ project & 4+ (20+) \\ \hline
 2 & Volvo Cars & Software Developer & 2 (3)\\ \hline
 3 & Volvo Cars & Director of Strategy and Concepts & 2+ (26) \\ \hline
 4 & Tier-1 supplier & Chief Engineer & 2.5 (17)\\ \hline
 5 & Tier-1 supplier & System Lead & 10+ (18) \\ \hline
 6 & Tier-1 supplier & Software Configuration Manager & 4\\ \hline
 7 & Tier-1 supplier & Open Source and Community Manager & $<$1 (28)\\ \hline
 8 & Tier-1 supplier & Chief Executive Officer & 10+\\ \hline
 9 & Tier-1 supplier & Product Manager & 3 (13)\\ \hline
 10 & Volvo Cars & Electrical System Architect & 2.5 (19)\\ \hline
 \end{tabular}
 \caption{Overview of the selected interviewees}
 \label{tab:population}
 \vspace{-.4cm}
 \end{table}

Also an appointment is made with the interviewee to verify and validate the transcript.

\noindent {\bf Transcribing:} To analyse the interview results, it is required to transcribe the interview in a clear and precise manner. The interviewee and interviewer speak English and the interview will be held in English, hence, this is also the case for the transcript. Also the analysis or coding is done in English.

\noindent {\bf Analysing:} For the analysis of the interview results, a protocol is applied to code the interview transcripts, retrieve useful information and analyze it for the research. This is done by using the method explained by Saldana \cite{saldana2015coding}. The transcript is divided into smaller and easier to code parts. This could be words, phrases, paragraphs or sections. The goal of this analysis is to find keywords or uncover themes that can be of value for the research. The findings are managed in the coding section of qualitative research tool, Atlas.ti\footnote{Qualitative Research tool Atlas.ti; https://atlasti.com}. This is done by constant comparison with the research, literature study and transcript in mind. After analysing the transcript and defining the preliminary codes, the final codes are defined based on relevance to the topics, importance, repetition or explicitly mentioned as important. The final codes are then used to draw conclusions. \todo{This needs to be rewritten to match the actual process}

\noindent {\bf Verifying:} Verification activities include ascertaining the validity, reliability and generalizability of the findings. {\em Validity} refers to the question if the study investigates what was originally intended to \cite{Kvale2015Interviews}. The purpose of the interviews is to gather from the interviewees insights and knowledge on the topics. The interview is semi-structured and the questions are intended as guidelines throughout the interview, so to safeguard validity. %The guidelines are not meant to be strict, so it allows also to meander to address other related topics. 
By applying the systematic and structured approach by Kvale and Brinkmann \cite{Kvale2015Interviews}, we safeguard the {\em reliability} of conducting, analyzing and reporting the interview process. Accordingly, the results are considered reliable too. As for {\em generalizability}, the results of the research can be applied to other domains than the automotive industry, e.g. aviation. The most advanced cars have more, or comparable, software than (fighter) airplanes\footnote{As said by Alfred Katzenbach, Director of IT management at Daimler: \url{http://spectrum.ieee.org/transportation/systems/this-car-runs-on-code}}.% Moreover, we assume that the results are general applicable with similar contexts. 

\noindent {\bf Reporting:} In this paper we report the results of the analysis and we answer the research questions RQ1, RQ2, and RQ3. 
%and RQ1b defined in the Introduction\todo{revise RQs, and should answer them all}. The results also support defining the Reference Model, which in turn will lay the foundation of the (intended) Impact Model~\cite{Blessing:2009:DDR:1717807}. The Impact Model will be of advisory nature and will describe the desired, improved situation that is expected. 
Even though the study is performed within a single OEM, the suppliers that work with Volvo Cars collaborate with many other OEMs. This gives the flavour that the results of our study will probably generalize to the automotive domain. Moreover, the way automotive OEMs work with suppliers is shared with many other domains, like for instance aviation. We then expect that the our results will be valuable also outside of the automotive domain.

%and this gives the These models\todo{plural?} are designed specifically for the domain of automotive software ecosystems. However, by abstracting the models it might be possible to apply them in other similar domains, such as aviation.

\subsection{Analysis protocol}
We conducted ten interviews with practitioners in a wide range of companies in the automotive software ecosystem. The interviews were transcribed and codified in Atlas.ti\footnote{Qualitative Research tool Atlas.ti; https://atlasti.com}. This tool is used to manage quotations and codes of transcripts. The quotations and codes describe the train of thoughts of the interviewees in order to process them into feasible datasets and to help supporting or negating propositions and research questions. All information that was relevant for the research was submitted as a quotation and connected to a code, including information that was not a direct answer to a research question or proposition. The codes were grouped into code groups to give an answer to one specific question, i.e. codes relevant to proposition 1 were submitted in code group ``Proposition 1". This is done to simplify the writing process of the findings for each individual research questions and propositions. This entire process made it easy to group all relevant quotations in an overview and to create a selective dataset. This dataset was then further analysed to retrieve the answers on the questions of every interviewee. In addition, the dataset provides extra information that could benefit the findings for the specific question. This could be extra background information, another perspective on the question or maybe start a new discussion. 

The results of this process are the findings, and the answers to the research questions and propositions.

\section{Results}\label{sec:results}


\subsection{Proposition 1: Increasing cross-organisational transparency of information is a necessary condition for cross-organisational CI\&D.}

{\bf .}

\subsection{Proposition 2: Increased cross-organisational transparency of information is considered positive.}

Increasing cross-organisational transparency can be perceived differently by different stakeholders. The interviewees were asked how they considered the increase of transparency across companies, and how they perceived business and personal relationships between companies. The complexity of the project is experienced as an impediment working against full transparency of information.

However, the interview results support the proposition, and the increase of information transparency is generally considered positive. This answer to proposition 2 is deduced from the following findings:

\noindent {\bf F2.1: Increased transparency is positive.} All the interviewees experience positively the increase of cross-organisational transparency between companies and employees, for both business and personal relationships between companies. The interviewees experience positive effects in terms of more awareness of the project status and increased mutual understanding. In particular, Volvo employees found sharing a workplace with Delphi developers especially effective for creating a shared mental model, thanks to the closer interaction with e.g. software developers.

\noindent {\bf F2.2: Trust is increased.} The increase of transparency of information increases trust, too, on two levels. Firstly, increased trust between stakeholders improves collaboration and communication. Secondly, the trust gained in past project (thanks to increased transparency of information) is more easily adopted in future projects as well. 

\subsection{Proposition 3: Cross-organisational sharing of information is considered simple between members of the projects.}

This proposition challenges the interviewees to critically evaluate the level of difficulty to share information across companies. In particular, the interviewees were asked about the difficulty to share information and the role of physical distance. The VCC/Delphi project is perceived as a complex project by both companies. Due to the complexity of this project, the project members experience that it is difficult to work against full transparency of information with other stakeholders. It is extremely difficult in the automotive industry to manage responsibilities on hardware and software (responsibility split) and managing intellectual property (IPR).

The findings partially support the proposition. The tooling and cross-organisational transparency, for example reducing physical distance, benefit cross-organisational information sharing, hence, reducing the complexity of a project. However, the automotive industry experience difficulty to share information, and manage responsibilities and IPR. This answer to proposition 3 is deduced from the following findings:

\noindent {\bf F3.1: Reducing physical distance most efficient.} The interviewees argued that reducing the physical distance between project members is the best situation for information sharing. The new way of working introduced in the VCC/Delphi project better enables the developers, but also management staff, to share information more efficient.

\noindent {\bf F3.2: Tooling support information sharing.} The tooling used for sharing information between developers or management staff is also a crucial factor for efficient information sharing across, but also within, companies. The interviewees from the VCC/Delphi project are positive about the tools and their supportive role in the project and agree that it reduces the complexity of the project.

\noindent {\bf F3.3: Managing responsibilities and IPR.} \pat{Add something about managing responsibilities and IPR}

\subsection{Proposition 4: Project members have sufficient information to perform their activities.}

This proposition challenges the interviewees to critically evaluate information, sent and received, between project members. They were asked what kind of information is (not) shared, if they have sufficient information available to perform their activities, and how this compares to other projects. All technical information is shared between project members, this includes source code, project information, and time planning. Commercial information is not shared between project members. This information contains strategic decisions, estimations, and third party agreements. The increased transparency across companies result in much more information than traditional projects, but equal or a bit more than agile projects. Third party agreements are an impediment for full transparency, because of licensing and responsibility issues.

However, the findings support this proposition and prove that project members have sufficient information available for their activities. This answer to proposition 4 is deduced from the following findings:

\noindent {\bf F4.1: Project members have sufficient information.} Both companies, VCC and Delphi, state that they have sufficient information available from both companies to perform their activities. 

\noindent {\bf F4.2: Lack overview in overall project.} The interviewees expressed that they miss an overview of their contribution in the product and overall project. Volvo hosted a presentation about the RFQ process and project details. This was perceived as positive by Delphi.


\subsection{Proposition 5: A more open transparency policy improves the quality of the project and its results.}

This proposition is developed to investigate whether the quality of the project results benefit from a more open transparency policy across companies. During the interview survey, the interviewees were asked a question about the effects of this policy on the project and its results. 

Findings emerging from the interviews are:

\noindent {\bf F5.1: High level of Quality.} The interviewees were unanimous about the positive effects on the quality of the project and its results. The quality improvements are already visible in the early stage of the project and they are confident about the improvements in the long term. An open transparency policy is positive for quality control because of mutual understanding of the project status and as a result, gaining efficiency. Delphi experiences healthy pressure on their activities and forces them to a certain quality level, because the customer is involved in the process of the product. To preserve the quality of the progress, Volvo is not heavily involved in the process.


Accordingly, we elicit the following possible answer to proposition 5:
The findings support this proposition and prove the quality of the project and its results improve due to a more open transparency policy across companies.

\subsection{Proposition 6: Strict contract-based collaboration is an impediment for cross-organisational Continuous Integration and Deployment}

This proposition challenges how practitioners experience a strict (or closed) contract in a cross-organisational setting where companies work together in software engineering projects using Continuous Integration and Deployment (CI\&D). During the interview survey, the interviewees were asked questions on the role of the contract when looking at information sharing and cross-organisational CI\&D. During the research it became clear that information sharing is seen as a crucial factor for cross-organisational collaboration. A closed contract regulates traditional project setups where the customer defines a list of requirements and the supplier has to fulfil it within a given time frame and budget. Originally the automotive industry is traditional and relatively closed. It however emerges that it changing towards greater cross-organisational transparency, participation in open source projects, and becoming a software-intensive sector. While still ongoing, this transition is confirmed by all interviewed stakeholders, and could lead to a cultural change toward adopting more open or agile contracts for cross-organisational CI\&D.

\begin{figure}[htb]
\centering
\includegraphics[width=\columnwidth]{figure/ss_CodeGroup6.png}
\caption{Cultural change toward adopting more open or agile contracts for cross-organisational CI\&D}
\label{fig:towardsAgile}
\end{figure}


Findings emerging from the interviews are:

\noindent {\bf F6.1: Agile contracts favour cross-organisational collaboration.} The interviewees share the opinion that a more open (or agile) contract would be healthier for the project and benefit cross-organisational collaboration. Although some of the project members from both customer and supplier companies are not fully aware of the contract details, they do share the feeling of being restricted. They believe that strict contracts conflict with an agile way of working instead of supporting it, and suggest to adopt more agile contracts, instead, also referred to as Time and Materials (T\&M) contracts\footnote{According to a T\&M contract, the contractor is being billed per hour regardless of the software project duration. If any additional features have to be developed the supplier charges just for the time spent by its employees working on a certain set of tasks [en.wikipedia.org]. This brings high flexibility to accommodate projects with evolving requirements, but also high uncertainty about the related costs.}. 
The interviewees agree that a T\&M contract allows for a better adaptation to project changes, distribution of resources, and it creates shared ownership otherwise hindered by closed contracts. They also argue in favour of a combination of a fixed price and T\&M contract, where stakeholders would agree on the product and cost estimation, but maintain high flexibility on how to produce it. This combination fulfils the need for flexibility and agility, but also the security for the customer. All interviewees made it clear that good collaboration between their companies is important from a legal and contractual perspective to support cross-organisational CI\&D.

\noindent {\bf F6.2: Closed contracts ease negotiation.} For a customer it is (still) more comfortable to work with closed contracts because one has more leverage and binds the supplier to pre-defined deliverables and deadlines. A Volvo manager involved in a RFQ projects, further states that it is hard for suppliers to negotiate with a T\&M or other agile contracts, and that closed contracts make it easier competing with other suppliers.

Accordingly, we elicit the following possible answers to proposition 6:

\begin{itemize}
\item Cross-organisational CI\&D benefits from a more open collaboration among companies, such as information sharing and adaptation to project changes. A strict contract is an impediment for this way of working related to cross-organisational CI\&D.
\item A strict contract is an impediment for cross-organisational CI\&D. Although there is no direct connection between a strict contract and cross-organisational CI\&D, the way of working related to this method benefits from an open or agile contract.
\item Cross-organisational CI\&D is possible with a strict contract, but synergy effects, i.e. collaboration and flexibility, are in effect when supported by an agile contract. By taking the synergy effects into account, a strict contract can be an impediment for Cross-Organisational CI\&D. An agile contract is difficult for a RFQ due to the open and uncertain characteristics of the contract.
\end{itemize}

\subsection{Proposition 7: If information is precise then, even though information is exchanged frequently, information overload is unlikely to be considered a problem.}

The assumption was that because of increased transparency between companies, information overload could occur. This proposition was developed to challenge project members how they experience information sharing. The interviewees were asked what kind of information is (not) shared, if it is much more or less information than other projects, and if they experienced information overload in their situation. Information precision is information sharing where supply and demand of information is synchronised. Information overload is information sharing where the receiving organisation receives more information than necessary. All the interviewees argued that information overload was not seen as a problem or risk.

Findings emerging from the interviews are:

\noindent {\bf F7.1: Understanding due collaboration.} 

\begin{itemize}
\item It is important to have an understanding, as result of clear collaboration and communication, of information needed by the receiving stakeholder, this prevents overreaction.
\item It is important to have an understanding of information needed by the receiving stakeholder, this prevents overreaction. This is achieved by clear collaboration and communication on what is necessary to share.
\end{itemize}

Accordingly, we elicit the following possible answer to proposition 7:

The findings support the proportion and information overload is not experienced as problem.



\subsection{Proposition 8: Standards and processes, based on industry-wide data and process standards benefit cross-organisational CI\&D.}

This proposition challenges the interviewees to experience the effects of industry-wide standards and processes in a cross-organisational setting where companies work together in software engineering projects using Continuous Integration and Deployment. The interviewees were asked if they use industry-wise standards or open source projects, and whether they find them beneficial for information sharing, which is important for cross-organisational CI\&D. The automotive industry is participating in more open source projects (i.e. AUTOSAR and GENIVI) and attempt to be good open source citizens. This development allows companies to hire new employees easier, because open source knowledge is more common than knowledge of proprietary technology.

Findings emerging from the interviews are:

%\begin{itemize}
%\item 
\noindent {\bf F8.1: Beneficial for information sharing.} The industry standards and open source projects allow a common language (i.e. AUTOSAR framework) and shared knowledge between project members, which makes it easier to communicate and share information.
%\item 

\noindent {\bf F8.2: Maturity and Management.} It is important for the success and adoption of open source projects and standards, that they are highly controlled by one person or organisation. According to several interviewees, the maturity is the crucial factor for the success or failure of an industry standard or open source project.
%\end{itemize}

Accordingly, we elicit the following possible answer to proposition 8:

Industry standards and open source projects allow a common language and shared knowledge, therefore, benefits information sharing, which is important for cross-organisational CI\&D.






% {\bf Industry Perspective.} 
% To preserve an open approach to the project, a product manager at a software development company suggests a combination of an agile and fixed price contract by creating project branches to avoid overhead in the main project.
%
%(Y)
% {\bf Maturity.} The automotive software ecosystem needs to adapt to the needs from the stakeholders. The industry is not mature enough, but is improving to adopt cross-organisational Continuous Integration and Deployment.
% Yes: Johnny Karlsson, Darrel Cullen, Lars Mattson, Petter Molder, Jacob Juul, Matti Larborn, 
% No: Anders Lindbom, Michael Svenstam, Mattias Almljum
\section{Discussion}\label{sec:discussion}

%\pat{How much upfront? What are the key artifacts that should be maintained?}


\subsection{Industry Perspective}

This study investigated a pilot project of Volvo Cars and one of its suppliers to investigate innovative ways of working and innovative collaboration models of the two organizations. Industries that are thinking or that are already in the process of opening their platforms towards an ecosystem perspective, or that simply are joining ecosystems might benefit from our findings.

%The proposition challenges interviewees to critically evaluate the impact of cross-organisational transparency on the business model strategy of their organisation. They were asked about the impact on the business model strategy on a functional and organisational level. 
%In general, the interviewees do not have an overview of the business model strategies of their organisation. 
In addition to the findings already discussed above, in this section we want to highlight that system-wide CI\&D could have an impact on both organizational and functional levels. 
%had a bigger impact in the early stage of the pilot project, but is getting less now the project is ongoing. 

%However, from our study it emerges that system-wide CI\&D %the interview results support the proposition, and cross-organisational transparency and CI\&D 
%could have an impact on business model strategies. %This answer to proposition 9 is deduced from the following findings:
%
%\noindent {\bf Impact on organisational level.} The organisational 
The impact on organisational level is perceived differently by the two companies companies. For Volvo Cars, it is perceived as one of the many projects and since they do not have the integration and end-responsibility for the product, it has less of an impact. However, for the supplier company, it is a high-risk experiment due to the complexity of the project. The employees of the supplier confirm that it has the highest complexity level the company accepted in its history.

%Th impact on functional level.} On functional level, 
The interviewees of both companies agree that the impacts on functional level of system-wide CI\&D %effects derived from the impact of cross-organisational transparency and CI\&D 
can be quantified in gain in efficiency, quality and time-to-market. Also CI\&D has less side effects than intermittent integration, in case of big bang integration.  

Finally, the interviewees state that system-wide CI\&D is not mature enough to see or foresees a big impact on business model strategies. 

\subsection{Academic Perspective}

Researchers that are interested in scaling agile methodologies to the entire organization or to inter-organizational collaborations might find in this paper interesting findings coming from an industrial project in a challenging domain that is living a profound transformation in these years. 

Researchers working on ecosystems can find in this paper an interesting example of ecosystem in the challenging automotive domain. Innovative ways to share information and knowledge within the ecosystem are needed. We also considered only one OEM and one supplier. We expect that different levels of sharing will be required according to degree of closeness to the OEM and the level of trust, and according on the purpose of the collaboration.

For what concerns contracts, our findings say that closed-contracts are an impediment to system-wide CI\&D, however they work well for what concern negotiation and IPR. Innovative contracts and/ot way of working are needed.

% {\bf Industry Perspective.} 
%  To preserve an open approach to the project, a product manager at a software development company suggests a combination of an agile and fixed price contract by creating project branches to avoid overhead in the main project.
% 
%  {\bf Maturity.} The automotive software ecosystem needs to adapt to the needs from the stakeholders. The industry is not mature enough, but is improving to adopt cross-organisational Continuous Integration and Deployment.
% % Yes: Johnny Karlsson, Darrel Cullen, Lars Mattson, Petter Molder, Jacob Juul, Matti Larborn, 
% % No: Anders Lindbom, Michael Svenstam, Mattias Almljum
\section{Related Works}\label{sec:relatedWorks}

\subsection{Continuous Integration and Deployment}


\subsection{Ecosystem and Transparency}
\section{Conclusion and Future Works}\label{sec:conclusion}

In this paper %\del{we investigated challenges and impediments of CI\&D that involves an OEM and one of its software suppliers.
%This is a relevant topic since automotive OEMs have strong motivations to embrace CI\&D because of pressures of reducing the time to market and improving flexibility while keeping quality. Automotive OEMs are increasingly becoming software companies and most of the software is often produced by external companies, i.e. Tier-1 and Tier-2 suppliers, that populate the automotive ecosystem. 
%For that reason, CI\&D in the automotive industry has to take into account also other actors of the ecosystem.  The increasing focus on software brings new challenges and opportunities for the software engineering field.}
%\chg{More specifically, in this paper}{
we report on a semi-structured interview study in the automotive industry. In particular, %} 
we investigated whether the way of working in automotive ecosystems should change while moving towards inter-organizational CI\&D; then we focused on {\em legal contracts} that regulate the agreements between OEMs and suppliers, and   
%are mainly based on closed contracts.
on {\em transparency} that is intended as the degree/level of information that is shared among the organisations collaborating in the same value-chain. %within the ecosystem. 
The study has been performed within \company{} in the context of a pilot and a large project that is researching new ways of working within the ecosystem. Employees of the supplier company seat within \company{} together with employees of the OEM and in some sense they work as a unique company, thus having more access to information. 
The results of the study show that more flexible contracts are needed, and that more transparency between OEM and suppliers is considered as an enabler for inter-organizational CI\&D. 
%
As future work we plan to investigate the challenges found in this paper within other companies and possibly in domains that are different from the automotive one. Another future research direction is investigating whether the new way of working will trigger new collaboration models within automotive domain, thus shifting towards a clearer and accepted win-win relationships between the actors of the ecosystem.

			

\balance

\bibliographystyle{IEEEtran}
\bibliography{bibliography}

\end{document}